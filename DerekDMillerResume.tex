% LaTeX resume using res.cls
\documentclass[overlapped]{res}
%\usepackage{helvetica} % uses helvetica postscript font (download helvetica.sty)
%\usepackage{newcent}   % uses new century schoolbook postscript font 
\usepackage[T1]{fontenc}
\usepackage{ae,aecompl}
\setlength{\textwidth}{5.5in} % set width of text portion

\begin{document}

% Center the name over the entire width of resume:
 \moveleft.5\hoffset\centerline{\large\bf Derek D. Miller}
% Draw a horizontal line the whole width of resume:
 \moveleft\hoffset\vbox{\hrule width\resumewidth height 1pt}\smallskip
% address begins here
% Again, the address lines must be centered over entire width of resume:
 \moveleft.5\hoffset\centerline{5109 Jacobs Creek Ct.}
 \moveleft.5\hoffset\centerline{Austin, TX 78749}
 \moveleft.5\hoffset\centerline{+1.512.695.7998}
 \moveleft.5\hoffset\centerline{derekdelmiller@gmail.com}


\begin{resume}
 
\section{Qualifications}  
\begin{itemize}
	\item Professional experience architecting and implementing security, secure boot, and content protection solutions for software, firmware, and SoC hardware
	\item Professional experience in Software Development in C, C++, perl, Go, and Python in Windows and Linux (gcc, Visual Studio, gdb, Valgrind)
	\item Professional experience in logic design and hardware validation (Verilog)
\end{itemize}
\section{Experience}
{\sl Principal Software Engineer} \hfill Jan 2016 to present \\
    Rubicon Labs
\begin{itemize}
    \item Designed and developed a key management web service in Go, exposed via a RESTful API
    \item Designed and implemented an Ansible deployment environment for the key
    management service which consisted of multiple interacting nodes on separate AWS
    EC2 instaces
    \item Developed requirements for contractor-implemented firmware and driver
    \item Heavily modified the opencryptoki open-source PKCS\#11 library for use with 
    custom hardware
    \item Developed simulation software emulating the hardware to enable testing of the
    opencryptoki modifications.
    \item Developed a multi-level test plan for each component of the project
\end{itemize}
{\sl Security Software Architect} \hfill July 2014 to December 2015 \\
	Amazon Web Services, Austin, TX
\begin{itemize}
    \item Developed Remote Procedure Call framework using Thrift-generated C and C++
	\item Architected, documented, and began implementation of a secure boot solution for an ARM-based platform that utilized X.509 certificates, with firmware, key, and certificate rollback protection
	\item Implemented the SHA256 cryptographic hashing algorithm for an ARMv8 platform using C and inline assembly
	\item Evaluated and selected cryptographic hardware IP from various vendors for inclusion into an SoC
	\item Architected the hardware that was needed to integrate the selected IP into a custom pipeline, including data flow and key management interfaces and memories
	\item Architected a hybrid hardware/software implementation of a secure crypto coprocessor
	\item Filed 9 patents, two issued so far
\end{itemize}
{\sl Security Software Architect} \hfill July 2012 to June 2014 \\
	Samsung System LSI, Austin, TX
\begin{itemize}
	\item Architected, documented, and oversaw implementation of a secure boot system for a new ARM-based platform from scratch that implemented ARM's Trusted Board Boot specification
	\item Implemented cryptographic functions (SHA256 and RSA) and an X.509 decoder for ROM and firmware
	\item Developed code that interfaced to a proprietary cryptographic accelerator and to key management hardware, including implementing RSA authentication utilizing an accelerator that utilized Montgomery Multiplication
	\item Architected, documented, and oversaw implementation of a solution for OpenSSL to utilize a proprietary cryptographic accelerator through an OpenSSL engine and a Linux kernel-mode driver
	\item Architected and documented a solution for TLS private key protection utilizing ARM's TrustZone technology
	\item Determined hardware requirements and provided consultations to the hardware team regarding design details and overall security strategy
	\item Participated in specification and benchmark development in several industry consortia, including UEFI, Trusted Computing Group, and EEMBC
\end{itemize}
{\sl Security Architect} \hfill September 2010 to June 2012 \\
	Intel IDG, Austin, TX
\begin{itemize}
	\item Designed and oversaw implementation of the security architecture of SoCs for the cellphone, tablet, and netbook markets
	\item Designed and oversaw implementation of media content protection hardware and software for SoCs in the cellphone, tablet and netbook markets
	\item Managed relationships between hardware and software teams in various locations (Texas, Oregon, China, India, Finland, Israel, Malaysia) 
	\item Managed IP supplier relationships and deliverables
\end{itemize}
{\sl Graphics Driver Software Developer} \hfill March 2008 to August 2010 \\
	Intel Visual Computing Group, Austin, TX
\begin{itemize}
	\item Designed and tested a Windows OpenGL driver for an experimental high-performance discrete graphics chip
	\item Designed and debugged elements of a high-performance, SIMD-based, software rasterizer
	\item Designed and debugged firmware for a multiprocessor, multi-threaded SIMD x86 architecture with texture sampling co-processors
	\begin{itemize}
		\item Code was a mixture of ANSI C and x86 assembly
	\end{itemize}
\end{itemize}
{\sl Component Design Engineering} \hfill May 2004 to March 2008 \\
	Intel Chipset Group, Austin, TX
\begin{itemize}
	\item Logic Designer (in Verilog) of cryptographic functions for integrated graphics chips, including OMAC and AES
	\item Developed functional simulators in C++ for HDCP, HDMI, memory paging systems, and various other graphics and media functions
	\item Developed an asynchronous interrupt validation methodology for the preemptive multitasking of a 3D graphics pipeline
\end{itemize}
{\sl CAD Engineer} \hfill December 2000 to May 2004 \\
	Intel Desktop Platforms Group, Austin, TX
\begin{itemize}
	\item Developed RTL simulators (C/C++, Verilog)
	\item Developed front-end design tools for semiconductor design (C/C++, perl, Verilog) utilizing both internal and vendor APIs
\end{itemize}
{\sl Software Developer} \hfill April 2000 to December 2000 \\
	Motorola Smartcard Solutions Division, Scottsdale, AZ
\begin{itemize}
	\item Developed encryption software for smartcard manufacturing using the IBM 4758 cryptography board and its API (C/C++)
\end{itemize}
{\sl Systems Engineer} \hfill April 1998 to April 2000 \\
	Motorola Satellite Communications Division, Chandler, AZ
\begin{itemize}
	\item Developed support and integration software for the Iridium satellite communications system (perl, C, C++)
	\item Developed satellite simulation software for a 77 satellite constellation, including calculations of the effects of Doppler shift on call completion rates
\end{itemize}
\section{Patents}
\begin{itemize}
    \item US Patent No. 9,479,340 - Controlling Use of Encryption Keys
    \item US Patent No. 9,674,162 - Updating Encrypted Cryptographic Key Pair
\end{itemize}
\section{Education} 
{\sl Master of Science,} Circuit Design, 4.0 GPA \hfill December 2006\\
                The University of Texas at Austin                 

{\sl Bachelor of Science,} Engineering Physics, 3.69 GPA \hfill December 1997\\
                The University of Oklahoma, Norman, OK\\
\end{resume}
\end{document}




