% LaTeX resume using res.cls
\documentclass[margin]{res}
%\usepackage{helvetica} % uses helvetica postscript font (download helvetica.sty)
%\usepackage{newcent}   % uses new century schoolbook postscript font 
\usepackage[T1]{fontenc}
\usepackage{ae,aecompl}
\setlength{\textwidth}{5.1in} % set width of text portion

\begin{document}

% Center the name over the entire width of resume:
 \moveleft.5\hoffset\centerline{\large\bf Derek D. Miller}
% Draw a horizontal line the whole width of resume:
 \moveleft\hoffset\vbox{\hrule width\resumewidth height 1pt}\smallskip
% address begins here
% Again, the address lines must be centered over entire width of resume:
 \moveleft.5\hoffset\centerline{5109 Jacobs Creek Ct.}
 \moveleft.5\hoffset\centerline{Austin, TX 78749}
 \moveleft.5\hoffset\centerline{+1.512.695.7998 or +1.512.670.8336}
 \moveleft.5\hoffset\centerline{derekdelmiller@gmail.com}


\begin{resume}
 
\section{Summary of Qualifications}  
\begin{itemize}
	\item Professional experience architecting security and content protection solutions for SoC hardware, software, and firmware
	\item Professional experience in Software Development in C, C++, and perl in Windows and Linux (g++, Visual Studio, gdb) -
 both tool software and system software
	\item Professional experience in Logic Design and hardware validation
\end{itemize}
\section{Professional Experience}
{\sl Security Software Architect} \hfill July 2012 to present \\
	Samsung System LSI, Austin, TX
\begin{itemize}
	\item Architected, documented, and oversaw implementation of a Secure Boot system for a new ARM-based platform from scratch that utilized X.509 certificates for firmware authentication
	\item Implemented cryptographic functions and an X.509 decoder for ROM and firmware
	\item Developed code that interfaced to proprietary cryptographic accelerator and key management hardware, including implementing RSA authentication utilizing an accelerator that utilized Montgomery Multiplication
	\item Architected, documented, and oversaw implementation of a solution for OpenSSL to utilize a proprietary cryptographic accelerator through an OpenSSL engine and a Linux kernel mode driver
	\item Architected, documented, and oversaw implementation of a solution for TLS private key protection utilizing ARM's TrustZone technology
	\item Determined hardware requirements and provided consultations to the hardware team regarding design details and overall hardware strategy
	\item Participated in specification and benchmark development in several industry consortia, including UEFI, Trusted Computing Group, and EEMBC
\end{itemize}
{\sl Security Architect} \hfill September 2010 to June 2012 \\
	Intel IDG, Austin, TX
\begin{itemize}
	\item Designed and oversaw implementation of security architecture of SoCs for the cellphone, tablet, and netbook markets
	\item Designed and oversaw implementation of media content protection hardware and software for SoCs in the cellphone, tablet and netbook markets
	\item Managed IP supplier relationships and deliverables
\end{itemize}
{\sl Graphics Driver Software Developer} \hfill March 2008 to August 2010 \\
	Intel Visual Computing Group, Austin, TX
\begin{itemize}
	\item Designed and tested a Windows 7 OpenGL driver for an experimental high-performance discrete graphics chip
	\item Designed and debugged elements of a software rasterizer
	\item Designed and debugged firmware for a multiprocessor, multi-threaded SIMD x86 architecture with texture sampling co-processors
	\begin{itemize}
		\item Code was a mixture of ANSI C and x86 assembly
	\end{itemize}
\end{itemize}
{\sl Component Design Engineering} \hfill May 2004 to March 2008 \\
	Intel Chipset Group, Austin, TX
\begin{itemize}
	\item Logic Designer (in Verilog) of cryptographic functions for integrated graphics chips, including OMAC and AES
	\item Developed functional simulators in C++ for HDCP, HDMI, memory paging systems, and various other graphics and media functions
	\item Developed an asynchronous interrupt validation methodology for the preemptive multitasking of a 3D graphics pipeline
\end{itemize}
{\sl CAD Engineer} \hfill December 2000 to May 2004 \\
	Intel Desktop Platforms Group, Austin, TX
\begin{itemize}
	\item Developed RTL simulators (C/C++, Verilog)
	\item Developed front-end design tools for semiconductor design (C/C++, perl, Verilog) utilizing both internal and vendor APIs
\end{itemize}
{\sl Software Developer} \hfill April 2000 to December 2000 \\
	Motorola Smartcard Solutions Division, Scottsdale, AZ
\begin{itemize}
	\item Developed encryption software for smartcard manufacturing using the IBM 4758 cryptography board and its API (C/C++)
\end{itemize}
{\sl Systems Engineer} \hfill April 1998 to April 2000 \\
	Motorola Satellite Communications Division, Chandler, AZ
\begin{itemize}
	\item Developed support and integration software for the Iridium satellite communications system (perl, C, C++)
	\item Developed satellite simulation software for a 77 satellite constellation, including calculations of the effects of Doppler shift on call completion rates
\end{itemize}
\section{EDUCATION} 
{\sl Master of Science,} Circuit Design, 4.0 GPA \hfill December 2006\\
                The University of Texas at Austin                 

{\sl Bachelor of Science,} Engineering Physics, 3.69 GPA \hfill December 1997\\
                The University of Oklahoma, Norman, OK\\
                Concentration: Computer Engineering\\
\end{resume}
\end{document}




